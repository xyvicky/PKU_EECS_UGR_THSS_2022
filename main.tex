\documentclass[UTF8,openany,AutoFakeBold,zihao=-4,screen]{ctexbook}
% ↑ 打印前去除screen选项,以应用奇偶页不对称(装订线5mm)的排版效果

\makeatletter
\newif\ifscreen
\@ifclasswith{ctexbook}{screen}{\screentrue}{\screenfalse}
\makeatother

\ifscreen
    \usepackage[a4paper,left=2.25cm,right=2.25cm,top=2.5cm,bottom=2.5cm]{geometry}
\else
    \usepackage[a4paper,left=2.5cm,right=2cm,top=2.5cm,bottom=2.5cm,twoside]{geometry}
\fi

\makeatletter
\def\cleardoublepage{
    \clearpage
    \ifscreen
        % do nothing
    \else
        \ifodd\c@page
            % do nothing
        \else
            \newpage
            \phantom{x} % one more empty page
            \clearpage
        \fi
    \fi
}
\makeatother

\usepackage{url}
\usepackage{calc}
\usepackage{amsmath}
\usepackage{bm}
\usepackage{amsfonts}
\usepackage{enumerate}
\usepackage{fancyhdr}
\usepackage[nottoc]{tocbibind}
\usepackage[super,square,sort,compress]{natbib}
\usepackage{multirow,booktabs,makecell}
\usepackage{graphicx}
\usepackage{datetime2}
\usepackage{tikz}
\usepackage[font=small,labelsep=space]{caption} %五号,宋体/Time new roman
%\renewcommand{\thetable}{\arabic{table}} %表格和图片编号不分章节,直接1,2,3 ...
%\renewcommand{\thefigure}{\arabic{figure}}
\renewcommand{\theequation}{\arabic{chapter}.\arabic{equation}} %公式标签 章.公式(均为阿拉伯数字)

\setmainfont[Ligatures=TeX]{Linux Libertine O}

\ifscreen
    \usepackage[colorlinks,allcolors=violet]{hyperref}
\else
    \usepackage[hidelinks]{hyperref}
\fi

\usepackage{enumitem}
\setlist{nosep}
\setlist[1]{leftmargin=3em}
\setlist[itemize]{label=\Large\textbullet}

% below: circled footnote symbols
\usepackage{pifont}
\usepackage[perpage,symbol*]{footmisc}
\DefineFNsymbols{circled}{{\ding{192}}{\ding{193}}{\ding{194}}{\ding{195}}{\ding{196}}{\ding{197}}{\ding{198}}{\ding{199}}{\ding{200}}{\ding{201}}}
\setfnsymbol{circled}
% below: normal number size in footnote
\makeatletter
\long\def\@makefntext#1{\noindent{{\@thefnmark}\enskip #1}}
\makeatother

\usepackage{tocloft} %自定义目录,说明中没有明确规定,和WORD自动生成目录格式一致

%“目录”两个字的格式
\renewcommand\cftbeforetoctitleskip{0pt}
\renewcommand\cftaftertoctitleskip{0pt}
\renewcommand\cfttoctitlefont{\bfseries\heiti\zihao{2}}

% below: toc styles

\cftsetpnumwidth{1em}

\renewcommand\cftchapfont{\heiti\small}
\renewcommand\cftchapleader{\cftdotfill{\cftchapdotsep}}
\renewcommand\cftchapdotsep{1}
\setlength{\cftchapnumwidth}{1em}
\renewcommand\cftchappagefont{\songti\small}
\renewcommand\cftbeforechapskip{0pt}

\renewcommand\cftsecfont{\songti\small}
\renewcommand\cftsecdotsep{1}
\setlength{\cftsecnumwidth}{1em}
\renewcommand\cftsecpagefont{\songti\small}
\renewcommand\cftsecindent{1em}
\renewcommand\cftbeforesecskip{0pt}

\renewcommand\cftsubsecfont{\songti\small}
\renewcommand\cftsubsecdotsep{1}
\setlength{\cftsubsecnumwidth}{1em}
\renewcommand\cftsubsecpagefont{\songti\small}
\renewcommand\cftsubsecindent{2em}
\renewcommand\cftbeforesubsecskip{0pt}

\renewcommand\cftsubsubsecfont{\songti\small}
\renewcommand\cftsubsubsecdotsep{1}
\setlength{\cftsubsubsecnumwidth}{1em}
\renewcommand\cftsubsubsecpagefont{\songti\small}
\renewcommand\cftsubsubsecindent{3em}
\renewcommand\cftbeforesubsubsecskip{0pt}

\usepackage{titlesec}%自定义章节标题
\ctexset{
    chapter = {
        format={\center \heiti \zihao {3}},
        beforeskip=0pt 
    }
}

\setcounter{tocdepth}{3}
\setcounter{secnumdepth}{3}
%使目录中有三级标题,即subsubsection

% below: section title fonts
\titleformat{\section}{\raggedright\zihao{3}\songti}{\thesection\quad}{0pt}{}
\titleformat{\subsection}{\raggedright\zihao{4}\songti}{\thesubsection\quad}{0pt}{}
\titleformat{\subsubsection}{\raggedright\zihao{-4}\songti}{\thesubsubsection\quad}{0pt}{}

\usepackage{xcoffins} % 用于设计封面格式
\usepackage{xcolor}


\input{head/ReviewTableHead}

\input{includes}
\newcommand{\chineseTitleA}{基于深度学习的}
\newcommand{\chineseTitleB}{壹壹肆伍壹肆壹玖壹玖捌壹零}
\newcommand{\englishTitleA}{Beyond P=NP: One One Four}
\newcommand{\englishTitleB}{Five One Four is All You Need}
\newcommand{\name}{唐可可}
\newcommand{\studentID}{1800099999}
\newcommand{\school}{信息科学技术学院(本科生学院)}
\newcommand{\major}{计算机科学与技术}
\newcommand{\advisor}{田中仁}
\newcommand{\advisorAffiliation}{偶像同好会}
\newcommand{\advisorTitle}{永远的神}
\newcommand{\paperGrade}{整挺好}

% 打印前请去除main.tex中documentclass的screen选项,以应用奇偶页不对称(装订线5mm)的排版效果
% 去除后将不再显示此内容
\newcommand{\buildInfo}{\center\colorbox{yellow}{build \today{} at \DTMcurrenttime{}, screen mode}}
% 或者可以这样保留screen选项并不显示此内容:
%\renewcommand{\buildInfo}{}

\newcommand{\advisorComment}{
% ↓ adjust total height below
\begin{minipage}[t][15cm][t]{\linewidth-2em}

\begin{center}
    \kaishu{(包含对论文的性质、难度、分量、综合训练等是否符合培养目标的目的等评价)}
\end{center}

\setlength{\parindent}{2em}

% https://zh.moegirl.org.cn/NEO_SKY,_NEO_MAP!

今日湛蓝的天空已不同于昨日,
明天的蓝天又会与今日不甚相同。
在你的眼中,也在我的眼中……
啊啊,已经无法用言语表述。

这种时候 如果你在我身边,
会松一口气呢。
所以,
所以明天也要抬头仰望,
在这里,和你一起。
自由地描绘这崭新的地图吧,
NEO MAP!

虽然还不知道要去往何方,
但那有趣的未来早已等待。
能有你与我一同欢笑,就会很快乐,
今天也感谢有你。

现在,就从现在开始,展开各自的地图。
轻松地飞奔而出吧。
梦想着,憧憬着,还想再次见证梦想,
想看到,真想看到呢。

NEO SKY, NEO MAP!
NEO SKY, NEO MAP!

\end{minipage}
}

\newcommand{\reviewDate}{\makebox[4em][c]{1919}年\makebox[2em][c]{8}月\makebox[2em][c]{10}日}
\newcommand{\honorDate}{\makebox[4em][c]{1145}年\makebox[2em][c]{1}月\makebox[2em][c]{4}日}


\title{}
\author{}
\date{}

\begin{document}

% no page numbers until main content (later)
\pagenumbering{gobble}

% 插入封面
% 声明需要的Coffin
\NewCoffin \result
\NewCoffin \topBox
\NewCoffin \pkuLogo
\NewCoffin \headingText
\NewCoffin \titleText
\NewCoffin \chineseTitleTextA
\NewCoffin \chineseTitleTextB
\NewCoffin \englishTitleTextA
\NewCoffin \englishTitleTextB
\NewCoffin \nameText
\NewCoffin \studentIDText
\NewCoffin \schoolText
\NewCoffin \majorText
\NewCoffin \advisorText
\NewCoffin \dateText
\NewCoffin \footnoteText

% 各个Coffin的内容
\SetHorizontalCoffin \result {}
\SetHorizontalCoffin \topBox {\color{white} \rule{210mm}{41mm}}
\SetHorizontalCoffin \pkuLogo {\includegraphics[width=83.9mm]{head/pku_logo}}
\SetVerticalCoffin \headingText{160mm}{\center\heiti\fontsize{36}{36}\textcolor{black}{本科生毕业论文}}
\SetVerticalCoffin \titleText{25.4mm}{\songti\zihao{3}{题目:}}
\SetVerticalCoffin \chineseTitleTextA{110mm}{\bfseries\heiti\zihao{2}\underline{\makebox[114mm][l]{\hfill \chineseTitleA \hfill}}}
\SetVerticalCoffin \chineseTitleTextB{110mm}{\bfseries\heiti\zihao{2}\underline{\makebox[114mm][l]{\hfill \chineseTitleB \hfill}}}
\SetVerticalCoffin \englishTitleTextA{110mm}{\bfseries\heiti\zihao{2}\underline{\makebox[114mm][l]{\hfill \englishTitleA \hfill}}}
\SetVerticalCoffin \englishTitleTextB{110mm}{\bfseries\heiti\zihao{2}\underline{\makebox[114mm][l]{\hfill \englishTitleB \hfill}}}
\SetVerticalCoffin \nameText{114mm}{\center\heiti\fontsize{16}{16}\textcolor{black}{姓\qquad\ \ \ 名:\underline{\makebox[86mm][c]{\fangsong{\name}}}}}
\SetVerticalCoffin \studentIDText{114mm}{\center\heiti\fontsize{16}{16}\textcolor{black}{学\qquad\ \ \ 号:\underline{\makebox[86mm][c]{\fangsong\zihao{-2}{\studentID}}}}}
\SetVerticalCoffin \schoolText{114mm}{\center\heiti\fontsize{16}{16}\textcolor{black}{院\qquad\ \ \ 系:\underline{\makebox[86mm][c]{\fangsong{\school}}}}}
\SetVerticalCoffin \majorText{114mm}{\center\heiti\fontsize{16}{16}\textcolor{black}{专\qquad\ \ \ 业:\underline{\makebox[86mm][c]{\fangsong{\major}}}}}
\SetVerticalCoffin \advisorText{114mm}{\center\heiti\fontsize{16}{16}\textcolor{black}{导师姓名:\underline{\makebox[86mm][c]{\fangsong{\advisor}}}}}
\SetVerticalCoffin \dateText{114mm}{\center\heiti\fontsize{18}{18}\textcolor{black}二\hspace{-2pt}\raisebox{-2pt}{\huge 〇}\hspace{-2pt}二二年六月}

% 打印前请去除main.tex中documentclass的screen选项,以应用奇偶页不对称(装订线5mm)的排版效果
% 去除后将不再显示黄框
\SetVerticalCoffin \footnoteText{114mm}{\buildInfo{}}

% 指定各个Coffin相对位置关系
\JoinCoffins \result \topBox
\JoinCoffins \result[\topBox-hc, \topBox-b] \pkuLogo[hc, b](0mm, -10.6mm)
\JoinCoffins \result[\topBox-hc, \topBox-b] \headingText[hc, b](0mm, -28.8mm)
\JoinCoffins \result[\headingText-hc, \headingText-b] \titleText[l, t](-77.25mm, -16mm)
\JoinCoffins \result[\headingText-hc, \headingText-b] \chineseTitleTextA[l, t](-62.85mm, -14.85mm)
\JoinCoffins \result[\headingText-hc, \headingText-b] \chineseTitleTextB[l, t](-62.85mm, -26.85mm)
\JoinCoffins \result[\headingText-hc, \headingText-b] \englishTitleTextA[l, t](-62.85mm, -38.85mm)
\JoinCoffins \result[\headingText-hc, \headingText-b] \englishTitleTextB[l, t](-62.85mm, -50.85mm)
\JoinCoffins \result[\headingText-hc, \headingText-b] \nameText[hc, t](0mm, -90mm)
\JoinCoffins \result[\nameText-hc, \nameText-b] \studentIDText[hc, t](0mm, 0mm)
\JoinCoffins \result[\studentIDText-hc, \studentIDText-b] \schoolText[hc, t](0mm, 0mm)
\JoinCoffins \result[\schoolText-hc, \schoolText-b] \majorText[hc, t](0mm, 0mm)
\JoinCoffins \result[\majorText-hc, \majorText-b] \advisorText[hc, t](0mm, 0mm)
\JoinCoffins \result[\advisorText-hc, \advisorText-b] \dateText[hc, t](0mm, -35mm)
\ifscreen
    \JoinCoffins \result[\advisorText-hc, \advisorText-b] \footnoteText[hc, t](0mm, -25mm)
\fi

% 输出封面
\thispagestyle{empty}
\ifscreen
    \newgeometry{left=0cm, bottom=0mm, top=0mm, right=0mm}
\else
    \newgeometry{left=2.5mm, bottom=0mm, top=0mm, right=0mm}
\fi
\noindent\TypesetCoffin \result
\restoregeometry
\cleardoublepage

% 插入导师评阅表
\thispagestyle{empty}
\renewcommand\arraystretch{1.2}

\begin{center}
{\songti\zihao{3}{北京大学本科毕业论文导师评阅表}}
\end{center}

\begin{table}[H]
	\centering
    \begin{tabular}{|rrrrrr|}
    \hline
    \multicolumn{1}{|p{4em}|}{学生姓名} & \multicolumn{1}{p{3em}|}{\name} & \multicolumn{1}{p{5em}|}{学生学号} & \multicolumn{1}{p{6em}|}{\studentID} & \multicolumn{1}{p{6em}|}{论文成绩} &  \multicolumn{1}{l|}{\paperGrade}\\
    \hline
    \multicolumn{1}{|l|}{学院(系)} & \multicolumn{3}{l|}{\school} & \multicolumn{1}{l|}{学生所在专业} & \multicolumn{1}{l|}{\major} \\
    \hline
    \multicolumn{1}{|l|}{导师姓名} & \multicolumn{1}{l|}{\advisor} & \multicolumn{1}{l|}{\makecell{导师单位/\\所在研究所}} & \multicolumn{1}{l|}{\advisorAffiliation} & \multicolumn{1}{l|}{导师职称} & \multicolumn{1}{l|}{\advisorTitle} \\
    \hline
    \multicolumn{2}{|c|}{\makecell{论文题目\\(中、英文)}} & \multicolumn{4}{c|}{\makecell{\chineseTitleA{}\chineseTitleB{}\\ \englishTitleA{} \englishTitleB{}}} \\
    \hline
    \multicolumn{6}{|c|}{导师评语} \\
    \multicolumn{6}{|c|}{\advisorComment{}} \\
    \multicolumn{6}{|r|}{
        导师签名:
        \begin{tikzpicture}[overlay]
            \node[anchor=south west,yshift=-1em] at (0,0) {
                \includegraphics[height=2em]{figures/sign-advisor}
            };
        \end{tikzpicture}
        \hspace{8em} 
    } \\
    \multicolumn{6}{|c|}{} \\
    \multicolumn{6}{|r|}{\reviewDate{}\hspace{2em} } \\
    \multicolumn{6}{|c|}{} \\
    \hline
    \end{tabular}
\end{table}

\renewcommand\arraystretch{1}
\cleardoublepage

\linespread{1.5}\selectfont
\chapter*{版权声明}
任何收存和保管本论文各种版本的单位和个人,未经本论文作者同意,不得将本论文转借他人,亦不得随意复制、抄录、拍照或以任何方式传播。否则,引起有碍作者著作权之问题,将可能承担法律责任。
\cleardoublepage

\normalsize
\linespread{1.5}\selectfont %小四号,宋体/Time new roman,1.5倍行距
\chapter*{\bfseries 摘要}

理解重复拍卖中学习算法的收敛性质是拍卖中的学习算法领域中的重要问题,并拥有广泛的应用场景,例如在线广告拍卖的场景。此文将聚焦于重复首价拍卖的场景,其中竞价者拥有不随时间改变的对物品的估值并通过基于均值的学习算法(mean-based learning algorithm)进行每一轮的报价。基于均值的学习算法包括一大类著名的无悔学习算法(no-regret learning algorithm),例如Multiplicative Weights Update和Follow the Perturbed Leader。我们完整刻画了这类基于均值的学习算法在我们的首价拍卖场景中的两种纳什收敛性,分别是(1)“时间平均纳什收敛”:竞价者采取单轮纳什均衡策略组合的轮数占总轮数的比例趋于1;(2)“末轮策略纳什收敛”:竞价者的策略组合趋于单轮纳什均衡。特别地,我们的结果将根据拥有相同最高估值的竞价者人数情况分为三类讨论:
\begin{itemize}
    \item 如果此人数至少有3人,竞价演化将几乎必然(almost surely)“时间平均纳什收敛”和“末轮策略纳什收敛”;
    \item 如果此人数有且仅有2人,竞价演化将几乎必然“时间平均纳什收敛”,但可能不“末轮策略纳什收敛”;
    \item 如果此人数有且仅有1人,竞价演化既可能不“时间平均纳什收敛”又可能不“末轮策略纳什收敛”。
\end{itemize}
此项研究的发现为学习算法收敛性的研究打开了新的思路。


\bigskip
\bigskip

关键词:重复首价拍卖,基于均值的学习算法,纳什均衡,收敛

\chapter*{\bfseries ABSTRACT}

%{\parindent0pt

Understanding the convergence properties of learning dynamics in repeated auctions is a timely and important question in the area of learning in auctions, with numerous applications in, e.g., online advertising markets. This work focuses on repeated first price auctions where bidders with fixed values for the item learn to bid using mean-based algorithms -- a large class of online learning algorithms that include popular no-regret algorithms such as Multiplicative Weights Update and Follow the Perturbed Leader. We completely characterize the learning dynamics of mean-based algorithms, in terms of convergence to a Nash equilibrium of the auction, in two senses: (1) \emph{time-average}: the fraction of rounds where bidders play a Nash equilibrium approaches 1 in the limit; (2) \emph{last-iterate}: the mixed strategy profile of bidders approaches a Nash equilibrium in the limit. Specifically, the results depend on the number of bidders with the highest value:
\begin{itemize}
    \item If the number is at least three, the bidding dynamics almost surely converges to a Nash equilibrium of the auction, both in time-average and in last-iterate.  
    \item If the number is two, the bidding dynamics almost surely converges to a Nash equilibrium in time-average but not necessarily in last-iterate.
    \item If the number is one, the bidding dynamics may not converge to a Nash equilibrium in time-average nor in last-iterate. 
\end{itemize}
Our discovery opens up new possibilities in the study of convergence dynamics of learning algorithms.

\bigskip
\bigskip

KEY WORDS: Repeated first price auction, Mean-based learning algorithm, Nash equilibrium, Convergence
%}
\clearpage

\chapter*{目录}
\renewcommand{\contentsname}{}
{\hypersetup{linkcolor=black} \tableofcontents}

% 正文页码从1开始,因此要确保前面是偶数页,不然奇偶页的边距就乱了
\cleardoublepage
\pagenumbering{arabic}
\setcounter{page}{1}

\normalsize
\linespread{1.5}\selectfont %正文,小四号,中文宋体,英文Time new roman,1.5倍行距

% below: page headers and footers

\fancypagestyle{plain} { % \chapter{} resets page style to plain for the first page
	\fancyhf{}
    \chead{\chineseTitleA\chineseTitleB}
	\cfoot{第\thepage{}页}
	\renewcommand{\headrulewidth}{0.7pt}
	\renewcommand{\footrulewidth}{0pt}
}
\fancypagestyle{pku} {
	\fancyhf{}
    \chead{\chineseTitleA\chineseTitleB}
	\cfoot{第\thepage{}页}
	\renewcommand{\headrulewidth}{0.7pt}
	\renewcommand{\footrulewidth}{0pt}
}

\pagestyle{pku}

\input{chapters/1_introduction}
\chapter{模型和预备知识}

教练我想学Java~\cite{jls8}。
\input{chapters/3_conclusion}
% [ ADD FOLLOWING CHAPTERS HERE ]

\small
\linespread{1}\selectfont
\bibliographystyle{bib/ACM-Reference-Format}
\bibliography{bib/base}

\linespread{1.5}\selectfont\normalsize %正文,小四号,中文宋体,英文Time new roman,1.5倍行距
\chapter*{致谢}
\addcontentsline{toc}{chapter}{致谢}
\input{chapters/90_ack}

\cleardoublepage
\chapter*{北京大学学位论文原创性声明和使用授权说明}
\addcontentsline{toc}{chapter}{北京大学学位论文原创性声明和使用授权说明}
\input{head/honor}
\thispagestyle{empty}

\end{document}
